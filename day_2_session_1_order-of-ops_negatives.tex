\documentclass[12pt]{beamer}
\usetheme{default} 

\setbeamertemplate{navigation symbols}{} %gets rid of navigation symbols
\setbeamertemplate{footline}{} %gets rid of bottom navigation bars
\setbeamertemplate{footline}[page number]{} %use this for page numbers

\setbeamertemplate{footline}{%
  \raisebox{5pt}{\makebox[\paperwidth]{\hfill\makebox[10pt]{\scriptsize\insertframenumber~~}}}}

\setbeamertemplate{itemize items}[circle] %round bullet points
\setlength\parskip{10pt} % white space between paragraphs

\usepackage{wrapfig}
\usepackage{subfig}
\usepackage{setspace}
\usepackage{enumerate}
\usepackage{graphicx}
\usepackage{amsmath}
\usepackage{amsfonts}
\usepackage{amssymb}
\usepackage{amsthm}
\usepackage[UKenglish]{isodate}
\usepackage{tikz}
\usepackage{natbib}
\def\checkmark{\tikz\fill[scale=0.4](0,.35) -- (.25,0) -- (1,.7) -- (.25,.15) -- cycle;} 

% bracketing shortcuts
\newcommand{\paren}[1]{\left(#1\right)}
\newcommand{\sqbracket}[1]{\left[#1\right]}
\newcommand{\cbracket}[1]{\left\{#1\right\}}
\newcommand{\abs}[1]{\left\lvert#1\right\rvert}
\newcommand{\norm}[1]{\left\lVert#1\right\rVert}
% set up the argmin operator, argmax
\DeclareMathOperator*{\argmin}{arg\,min}
\DeclareMathOperator*{\argmax}{arg\,max}

\newcommand{\myframe}[1]{\begin{frame} \frametitle{#1}}
% the preamble
\title{Day 2, Session 1: Order of operations and negative numbers}
\author{Brian Williamson}
\institute{EPI/BIOST Bootcamp 2016}
\date{26 September 2016}

% Start the document
\begin{document}
% The title page
\begin{frame}
\titlepage
\end{frame}

\section{Order of operations}
\myframe{Evaluating expressions}
\begin{itemize}
\item Example expression: $3(1 + 2) + 5$
\item[]
\item How do we evaluate the above expression? In other words:
\begin{itemize}
\item Which terms to we compute first?
\item[]
\item Are there rules for evaluating expressions?
\end{itemize}
\end{itemize}
\end{frame}

\myframe{Order of operations}
\begin{itemize}
\item Rules for evaluating expressions:
\begin{enumerate}
\item Parentheses
\item Exponents
\item Multiplication and division
\item Addition and subtraction
\end{enumerate}
\item[]
\item A handy memory device: PEMDAS --- Please Excuse My Dear Aunt Sally
\end{itemize}
\end{frame}

\myframe{Example 1: order of operations in action!}
\begin{itemize}
\item Example from slide 2: $3(1 + 2) + 5$
\item[]
\item This notation is equivalent to $3\times (1+2) + 5$
\item[]
\item Apply PEMDAS: \hfill \underline{Current Expression}
\begin{enumerate}
\item Parentheses: add 1 and 2 \hfill $3(3) + 5$
\item Exponents: none \hfill $3(3) + 5$
\item Multiplication: multiply 3 and 3 \hfill $9 + 5$
\item Division: none \hfill $9 + 5$
\item Addition: add 9 and 5 \hfill 14
\item Subtraction: none
\end{enumerate}
\item[]
\item The final answer is 14!
\end{itemize}
\end{frame}

\myframe{Example 2: order of operations with exponents!}
\begin{itemize}
\item Expression: $\dfrac{(2^2 + 5)^2}{3\times 3} + 5$
\item[]
\item Apply PEMDAS: \hfill \underline{Current Expression}
\begin{enumerate}
\item Parentheses: $2^2 + 5$. \hfill $\frac{(2^2 + 5)^2}{3\times 3} + 5$
\item[$\cdot$] Need to apply PEMDAS again!
\begin{enumerate}
\item Parentheses: none \hfill $\frac{(2^2 + 5)^2}{3\times 3} + 5$
\item Exponents: $2^2 = 4$ \hfill $\frac{(4 + 5)^2}{3\times 3} + 5$
\item Multiplication/division: none \hfill $\frac{(4 + 5)^2}{3\times 3} + 5$
\item Addition/subtraction: $4 + 5 = 9$ \hfill $\frac{(9)^2}{3\times 3} + 5$
\end{enumerate}
\item Exponents: $9^2 = 81$ \hfill $\frac{81}{3\times 3} + 5$
\item Multiplication: $3 \times 3 = 9$ \hfill $\frac{81}{9} + 5$
\item Division: $81/9 = 9$ \hfill $9 + 5$
\item Addition/subtraction: $9 + 5 = 14$!
\end{enumerate}
\end{itemize}
\end{frame}

\section{Negative numbers}


\end{document}
